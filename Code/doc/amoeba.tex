\documentclass[12pt]{article}
\title{Amoeba cartoon application for diffusion in multiple volumes}
\author{D. T. Graves, T. J. Ligocki}
\begin{document}
\maketitle

\section{Algorithm summary}

Amoeba solves the variable coefficient diffusion equation
$$
\frac{\partial \phi}{\partial t} = \nabla \cdot \nu \nabla \phi + \rho
$$
where $\phi$ can have any number of variables and both $\nu$ (the diffusion
coefficient) and $\rho$ (the source term) are functions of both space
and time.   Boundary condtions are set to homogeneous Neumann at the
domain boundary.   Boundary condtions at embedded boundaries are set
to a data value at each irregular face.   This data value is reset at
every step and can be a function of time and space and the solution. 
Space is divided into a collection of connected volumes
${\cal C}$.   The algorithm is summarized as follows.  At each time
step we:
\begin{itemize}
\item Set the source term and the boundary conditions at each
  irregular cell.
\item Advance the solution using backward Euler one time step for each
  component and each volume (this involves $\sum_{i=1}^{N_v} c_i$ solves,
  where $N_v$ is the number of volumes and $c_i$ is the number of components
  in volume $i$).
\end{itemize}

\section{Current state of play}
\begin{itemize}
\item Amoeba runs in two and three dimensions. 
\item It runs slowly.
\item The solvers usually converge but sometimes a volume will not
  converge.
\item I have tried problems with 2 volumes (in {\tt debug.inputs}) and
9 volumes (in {\tt multisphere.inputs}).
\end{itemize}
\end{document}
